\documentclass{article}

%some useful packages
\usepackage{amsmath,amsthm,amsfonts,amssymb}  %math stuff
\usepackage{graphicx} %to embed images
\usepackage{enumitem} %fancy lists
\usepackage{fullpage} %smaller margins
\usepackage{color} %smaller margins
\usepackage{float}
\usepackage{array} % For better table control
\usepackage{stackrel}
\usepackage{nicematrix}
\NiceMatrixOptions{
code-for-first-row = \color{blue} ,
code-for-first-row = \color{blue} ,
code-for-last-row = \color{blue} ,
code-for-first-col = \color{blue} ,
code-for-last-col = \color{blue}
}


\title{Math 462 Project}
\author{Authors:}
\date{February 2025}

\begin{document}

\maketitle

\section{Introduction}
The numbers of Notch (\(N_P\)) and Delta (\(D_P\)) molecules changing over time due to random reactions motivates us to employ a stochastic model. Each reaction happens with a certain probability per unit time. \\ 
However, we want to show that, in the large-number limit, the expected values \(E[N_P]\) and \(E[D_P]\) follow the deterministic ODEs:

\[
\frac{dN_P}{dt} = F(D_{\text{neighbour},P}) - \mu N_P
\]
\[
\frac{dD_P}{dt} = G(N_P) - \rho D_P
\]

\hfill

We establish a detailed stochastic model for Notch-Delta signalling, ensuring that it converges to the deterministic ODE model under suitable regularity conditions. The system is described by a continuous-time Markov process (CTMC) and then derive the expected value equations to show how it approximates the deterministic model.

\section{Stochastic Model for Notch-Delta Signalling}
\subsection{Chemical Reactions and Propensities}
To construct a probabilistic model, we define the stochastic dynamics of the Delta-Notch interaction as a set of discrete reaction events. We introduce the following reactions:

\begin{align*}
  N_{P} + D_{P} &\xrightarrow{k_1} C_{P} \quad \text{(Binding of Notch and Delta)} \\
  C_{P} &\xrightarrow{k_2} N_{P}^{*} + D_{P} \quad \text{(Activation of Notch)} \\
  C_{P} &\xrightarrow{k_{-1}} N_{P} + D_{P} \quad \text{(Dissociation of the complex)}
\end{align*}

where:

\begin{itemize}
  \item $C_{P}$ represents the Notch-Delta complex in cell $P$.
  \item $k_1$ is the rate at which Notch and Delta bind to form a complex.
  \item $k_2$ is the rate at which the complex activates Notch, producing an intracellular signal.
  \item $k_{-1}$ is the dissociation rate of the complex back into Notch and Delta.
\end{itemize}

Reactions describing Notch and Delta production and degradation  are summarized in the table below:

\begin{table}[H]
    \centering
    \begin{tabular}{|c|c|c|} \hline 
         Reaction&  Reaction Equation& Associated term\\ \hline 
         Notch production&  \( N_P \to N_P + 1 \)& \( F(D_{\text{neighbour},P}) \)\\ \hline 
         Notch degradation&  \( N_P \to N_P - 1 \)& \( \mu N_P \)\\ \hline 
         Delta production&  \( D_P \to D_P + 1 \)& \( G(N_P) \)\\ \hline 
         Delta degradation&  \( D_P \to D_P - 1 \)& \( \rho D_P \)\\\hline
    \end{tabular}
    \caption{A summary of Notch and Delta production and degradation reactions}
    \label{1}
    
\end{table}

where:
\begin{itemize}
    \item \( F(D_{\text{neighbour},P}) \) is an increasing function describing the effect of Delta on Notch production.
    \item \( G(N_P) \) is a decreasing function representing Notch inhibiting Delta.
\end{itemize}

\subsection{Chemical Master Equation and State Probabilities}
Let \( P(N_P, D_P, C_P, t) \) be the probability that the system is in state \( (N_P, D_P, C_P) \) at time \( t \). The probability evolves according to the chemical master equation (CME):

\[
\frac{dP}{dt} = \sum_{\text{all reactions}} \Big[ \text{gain term} - \text{loss term} \Big]
\]

\begin{align*}
  \frac{dP(N_P, D_P, C_P, t)}{dt} &= k_1 (N_P - 1)(D_P - 1) P(N_P - 1, D_P - 1, C_P + 1, t) \\
  &\quad + k_{-1} (C_P + 1) P(N_P + 1, D_P + 1, C_P - 1, t) \\
  &\quad + k_2 (C_P + 1) P(N_P, D_P, C_P - 1, t) \\
  &\quad + F(D_{\text{neighbour},P}) P(N_P - 1, D_P, C_P, t) \\
  &\quad + G(N_P) P(N_P, D_P - 1, C_P, t) \\
  &\quad + \mu (N_P + 1) P(N_P + 1, D_P, C_P, t) \\
  &\quad + \rho (D_P + 1) P(N_P, D_P + 1, C_P, t) \\
  &\quad - (k_1 N_P D_P + k_{-1} C_P + k_2 C_P + \mu N_P + \rho D_P) P(N_P, D_P, C_P, t).
\end{align*}

\subsection{Expected Value Approximation}
To derive the deterministic equations, we compute the expected values:

\[
E[N_P] = \sum_{N_P} N_P P(N_P, D_P, C_P, t)
\]

\[
E[D_P] = \sum_{D_P} D_P P(N_P, D_P, C_P, t)
\]

\[
E[C_P] = \sum_{C_P} C_P P(N_P, D_P, C_P, t)
\]

Taking the time derivative:

\[
\frac{dE[N_P]}{dt} = \sum_{N_P} N_P \frac{dP(N_P, D_P, C_P, t)}{dt}.
\]

Substituting from the CME, we obtain:

\[
\frac{dE[N_P]}{dt} = -k_1 E[N_P]E[D_P] + k_{-1}E[C_P] + k_2 E[C_P] + F(E[D_{\text{neighbour},P}]) - \mu E[N_P].
\]

Similarly, for Delta:

\[
\frac{dE[D_P]}{dt} = -k_1 E[N_P]E[D_P] + k_{-1}E[C_P] + G(E[N_P]) - \rho E[D_P].
\]

For Complexes, we start with the transition terms in the CME.
\begin{align*}
\frac{dP(N_P, D_P, C_P, t)}{dt} = &+ 
 k_1 (N_P - 1)(D_P - 1) P(N_P - 1, D_P - 1, C_P + 1, t)\\
& + k_{-1} (C_P + 1) P(N_P + 1, D_P + 1, C_P - 1, t)\\
& + k_2 (C_P + 1) P(N_P, D_P, C_P - 1, t)
\end{align*}

We compute the expected value for complexes, \( E[C_P] \), and substitute into the time derivative expression:

\[
\frac{dE[C_P]}{dt} = k_1 E[N_P] E[D_P] - (k_{-1} + k_2) E[C_P] - k_{-1}
\]

Rearranging, we get the final form:
\[
\frac{dE[C_P]}{dt} = k_1 E[N_P]E[D_P] - (k_{-1} + k_2) E[C_P].
\]

Using the main equation derived above, we substitute \( dP/dt \) for each reaction.

For Notch production, we expand:

\[
\sum_{N_P} N_P \left[ F(D_{\text{neighbour},P}) P(N_P - 1, D_P, C_P, t) - F(D_{\text{neighbour},P}) P(N_P, D_P, C_P, t) \right]
\]

Shifting the sum over \( N_P \) gives:

\[
\sum_{N_P} (N_P + 1) F(D_{\text{neighbour},P}) P(N_P, D_P, C_P, t) - \sum_{N_P} N_P F(D_{\text{neighbour},P}) P(N_P, D_P, C_P, t)
\]

Rearranging:

\[
\sum_{N_P} F(D_{\text{neighbour},P}) P(N_P, D_P, C_P, t) = E[F(D_{\text{neighbour},P})]
\]

Thus, the equation simplifies to:

\[
\frac{dE[N_P]}{dt} = E[F(D_{\text{neighbour},P})] - \mu E[N_P]
\]

Similarly, for Delta:

\[
\frac{dE[D_P]}{dt} = E[G(N_P)] - \rho E[D_P]
\]

\subsection{The Mean-Field Approximation}
In general, expected values of nonlinear functions like \( E[F(D_P)] \) are not necessarily equal to \( F(E[D_P]) \). However, under large-number assumptions, fluctuations around the mean become small (due to the law of large numbers), and we approximate:

\[
E[F(D_{\text{neighbour},P})] \approx F(E[D_{\text{neighbour},P}])
\]

\[
E[G(N_P)] \approx G(E[N_P])
\]

This assumes that each individual cell’s fluctuations average out in large populations.

\subsection{Deterministic Model Recovery}

Substituting these approximations into our equations:

\[
\frac{dE[N_P]}{dt} = F(E[D_{\text{neighbour},P}]) - \mu E[N_P]
\]

\[
\frac{dE[D_P]}{dt} = G(E[N_P]) - \rho E[D_P]
\]

Thus, we recover the deterministic system:

\[
\frac{dN_P}{dt} = F(D_{\text{neighbour},P}) - \mu N_P
\]

\[
\frac{dD_P}{dt} = G(N_P) - \rho D_P
\]

The calculation confirms that in the large-number limit, the stochastic model converges to the deterministic ordinary differential equation (ODE) system. This implies that as the molecular count increases, transitioning the system to a more macroscopic scale, the stochastic model naturally aligns with the deterministic framework.

\subsection{Convergence to the Deterministic Model}

To show convergence, define fluctuations:

\[
X = N_P - E[N_P], \quad Y = D_P - E[D_P], \quad Z = C_P - E[C_P].
\]

Taking expectations:

\[
\frac{dE[X]}{dt} = -\mu E[X], \quad \frac{dE[Y]}{dt} = -\rho E[Y], \quad \frac{dE[Z]}{dt} = -(k_{-1} + k_2)E[Z].
\]

These decay exponentially, meaning that as time progresses:

\[
E[X], E[Y], E[Z] \to 0.
\]

Thus, the stochastic model converges to the deterministic ODE model as molecule numbers increase.


\end{document}
