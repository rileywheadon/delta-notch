\documentclass{article}

\usepackage[margin=1.5in]{geometry}
\usepackage{amsmath}

\begin{document}

\begin{flushleft}

\section{Collier et al., 1996}

Collier et al. make the following five modelling assumptions:
\begin{enumerate}
  \item Cells interact through Delta-Notch signalling if and only if they are adjacent.
  \item Production of Notch is an increasing (hill) function of Delta in neighbouring cells.
  \item Production of Delta is a decreasing (hill) function of Notch in the same cell.
  \item Production of Notch and Delta is balanced by decay proportional to concentration.
  \item Low levels of Notch cause the primary fate, high levels cause the secondary fate.
\end{enumerate}

Define the following variables:

\begin{itemize}
  \item $\tau$: time
  \item $N_{P}$: notch activity (concentration) in cell $P$
  \item $D_{P}$: delta activity (concentration) in cell $P$
  \item $\overline{D}_{P}$: average delta activity in neighbours of $P$
  \item $N_{0}$: typical notch activity (across all cells)
  \item $D_{0}$: typical delta activity (across all cells)
  \item $\mu$: The decay rate for notch (assumed constant)
  \item $\rho$: The decay rate for delta (assumed constant)
\end{itemize}

Then, define the following system of differential equations, where $F:[0, \infty) \rightarrow [0, \infty)$ is a continuous increasing function and $G: [0, \infty)\rightarrow [0, \infty)$ is a continuous decreasing function.

$$
\begin{aligned}
  \frac{d(N_{P} / N_{0})}{d\tau} = F(\overline{D}_{P} / D_{0}) - \mu N_{P} / N_{0} \\[5pt]
  \frac{d(D_{P} / D_{0})}{d\tau} = G(N_{P} / N_{0}) - \rho D_{P} / D_{0}
\end{aligned}
$$

This equation is inherently nondimensional, since $N_{P} / N_{0}$ and $D_{P} / D_{0}$ have no units.

\subsection{Writing Down a Dimensional Model}

Let us continue to use $N_{P}$, $D_{P}$, and $\overline{D}_{P}$ as defined above. Let $t$ denote the dimensional time and let $u$ and $r$ be the dimensional decay rates for notch and delta. Then, assuming the dimensional analogues of $F$ and $G$ are hill functions of order $1$, we have:

$$
\begin{aligned}
  \frac{dN_{P}}{dt} &= \frac{\overline{D}_{P}}{d_{0} + \overline{D}_{P}} - u N_{P} \\[5pt]
  \frac{dD_{P}}{dt} &= \frac{n_{0}}{n_{0} + N_{P}} - rD_{P}
\end{aligned}
$$

In the equation above, $d_{0}$ is the half-max of notch production induced by delta in neighbouring cells while $n_{0}$ is the half-min of delta production induced by notch. 

\medskip

This model is a good start, but we will need to explicitly model the concentration of delta-notch complexes $C$ in order to write down a probabilistic model. Since $D_{P}$ and $N_{P}$ denote the level of delta and notch \emph{activation} (that is, the concentration of complexes), 

\subsection{Deriving a Probabilistic Model}

TBD

\end{flushleft}

\end{document}
