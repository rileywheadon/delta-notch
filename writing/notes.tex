\documentclass{article}

\usepackage[left=1.5in, right=1.5in, top=1in, bottom=1in]{geometry}
\usepackage{amsmath}
\usepackage{amssymb}

\begin{document}

\begin{flushleft}

\section{Collier et al., 1996}

Collier et al. make the following five modelling assumptions:
\begin{enumerate}
  \item Cells interact through Delta-Notch signalling if and only if they are adjacent.
  \item Production of Notch is an increasing (hill) function of Delta in neighbouring cells.
  \item Production of Delta is a decreasing (hill) function of Notch in the same cell.
  \item Production of Notch and Delta is balanced by decay proportional to concentration.
  \item Low levels of Notch cause the primary fate, high levels cause the secondary fate.
\end{enumerate}

Define the following variables:

\begin{itemize}
  \item $\tau$: time
  \item $N_{P}$: notch activity (concentration) in cell $P$
  \item $D_{P}$: delta activity (concentration) in cell $P$
  \item $\overline{D}_{P}$: average delta activity in neighbours of $P$
  \item $N_{0}$: typical notch activity (across all cells)
  \item $D_{0}$: typical delta activity (across all cells)
  \item $\mu$: The decay rate for notch (assumed constant)
  \item $\rho$: The decay rate for delta (assumed constant)
\end{itemize}

Then, define the following system of differential equations, where $F:[0, \infty) \rightarrow [0, \infty)$ is a continuous increasing function and $G: [0, \infty)\rightarrow [0, \infty)$ is a continuous decreasing function.

$$
\begin{aligned}
  \frac{d(N_{P} / N_{0})}{d\tau} = F(\overline{D}_{P} / D_{0}) - \mu N_{P} / N_{0} \\[5pt]
  \frac{d(D_{P} / D_{0})}{d\tau} = G(N_{P} / N_{0}) - \rho D_{P} / D_{0}
\end{aligned}
$$

This equation is inherently nondimensional, since $N_{P} / N_{0}$ and $D_{P} / D_{0}$ have no units.

\medskip

\textbf{Remark:} The notch activity can also be taken to be the number of complexes formed by the binding of notch to its activated ligand delta (Collier et al. 1996).

\newpage

\subsection{Deriving a Probabilistic Model}

We will make the assumption that each cell has $N$ notch molecules (which can be in a complex, or not). As we did above, define $\overline{D}$ to be the average number of active delta molecules in neighbouring cells. Let $k_{b}$ be the binding rate and $k_{u}$ the unbinding rate. Let $p_{n}(t)$ be the probability there are $n$ complexes at time $t$. Then, define the Kolmogorov forward equation:

$$
\begin{aligned}
  p_{n}(t + \Delta t)= &\,\,p_{n-1}(t) \cdot \underbrace{(N - (n - 1))(\overline{D} - (n - 1)) k_{b} \Delta t}_{\text{P(binding)}}  \\[5pt]
                      &+ p_{n}(t) \cdot \underbrace{(1 - (N - n)(\overline{D} - n)k_{b}\Delta t)}_{\text{P(no binding)}} \cdot \underbrace{(1 - n k_{u} \Delta t)}_{\text{P(no unbinding)}} \\[5pt]
                      &+ p_{n+1}(t) \cdot \underbrace{(n + 1)k_{u}\Delta t}_{\text{P(unbinding)}}
\end{aligned}
$$

Rearranging this equation (and cancelling out the term with $\Delta t^2$) we get: 

$$
\begin{aligned}
p_{n}'(t) = &\,\,p_{n-1}(t) \cdot (N - (n - 1))(\overline{D} - (n - 1)) k_{b} \\[5pt]
            &-p_{n}(t)\cdot (N-n)(\overline{D} - n)k_{b} - p_{n}(t) \cdot n k_{u} \\[5pt]
            &+ p_{n+1}(t) \cdot (n + 1)k_{u}
\end{aligned}
$$

Let $C(t)$ be the number of complexes at time $t$. From the equation above, we have:

$$
y(t) = \mathbb{E}[C(t)] = \sum_{n = 0}^{N} np_{n}(t) \Rightarrow y'(t) = \sum_{n = 0}^{N} np_{n}'(t)
$$

Now, we substitute $p_{n}'(t)$ for the expression derived above:

$$
\begin{aligned}
  y'(t) = &\,\,\sum_{n = 2}^{N}  np_{n-1}(t)(N - (n - 1))(\overline{D} - (n - 1)) k_{b} \\[5pt]
          &-  \sum_{n = 1}^{N - 1} np_{n}(t) (N - n)(\overline{D} - n )k_{b} -  \sum_{n = 1}^{N}  n^2 p_{n}(t) k_{u} \\[5pt]
          &- \sum_{n = 1}^{N - 1}  np_{n+1}(t) (n+1)k_{u}
\end{aligned}
$$

We will need to reindex the $p_{n-1}$ and $p_{n+1}$ terms. To do this, begin by rewriting:

$$
\begin{aligned}
  y'(t) = &\,\,\sum_{n = 2}^{N}  (n-1 + 1)p_{n-1}(t)(N - (n - 1))(\overline{D} - (n - 1)) k_{b} \\[5pt] 
          &-  \sum_{n = 1}^{N - 1} np_{n}(t) (N - n)( \overline{D} - n)k_{b} -  \sum_{n = 1}^{N}  n^2 p_{n}(t) k_{u} \\[5pt]
          &- \sum_{n = 1}^{N - 1}  (n + 1 - 1)p_{n+1}(t) (n+1)k_{u}
\end{aligned}
$$

Now, perform reindexing on the $p_{n- 1}$ and $p_{n+1}$ terms:

$$
\begin{aligned}
y'(t) = &\,\,\sum_{n = 1}^{N - 1} (n + 1)p_{n}(t)(N - n) (\overline{D} - n)k_{b} \\[5pt]
        &-  \sum_{n = 1}^{N - 1} np_{n}(t) (N - n)(\overline{D} - n)k_{b} -  \sum_{n = 1}^{N}  n^2 p_{n}(t) k_{u} \\[5pt]
        &+  \sum_{n = 2}^{N}  (n - 1 )p_{n}(t) nk_{u} 
\end{aligned}
$$

Now, we cancel out like terms to get the following:

$$
y'(t) = \sum_{n = 1}^{N} p_{n}(t)(N - n)(\overline{D} - n)k_{b} - \sum_{n = 1}^{N} np_{n}(t)  k_{u}
$$

Therefore, 

$$
y'(t) = (N - y)(\overline{D} - y)k_{b} -  yk_{u}
$$

Note that the $(N - y)(\overline{D} - y)$ term is an increasing function of $\overline{D}$, which means that it satisfies the conditions imposed by the Collier et al. (1996). However, it is not a hill function, which means the authors made different assumptions about the system.

\end{flushleft}

\end{document}
